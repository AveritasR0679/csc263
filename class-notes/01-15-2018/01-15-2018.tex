\documentclass[12pt]{article} 
\usepackage[letterpaper, margin=0.5in]{geometry}        		
\geometry{letterpaper}
\usepackage[parfill]{parskip} 
\usepackage{framed}
\usepackage{graphicx}
\usepackage{amsmath}
\usepackage{amssymb}
\usepackage{qtree}
\usepackage{makecell}

\usepackage{mathtools}
\DeclarePairedDelimiter\ceil{\lceil}{\rceil}
\DeclarePairedDelimiter\floor{\lfloor}{\rfloor}
\DeclarePairedDelimiter\abs{\lvert}{\rvert}%
\DeclarePairedDelimiter\norm{\lVert}{\rVert}%
    % \abs & \norm resizes brackets, starred version doesn't
    \makeatletter
    \let\oldabs\abs
    \def\abs{\@ifstar{\oldabs}{\oldabs*}}
    %
    \let\oldnorm\norm
    \def\norm{\@ifstar{\oldnorm}{\oldnorm*}}
    \makeatother
    
\newcommand\tab[1][0.25cm]{\hspace*{#1}}
\newcommand\imp{\rightarrow}
\newcommand\thfr{\tab \therefore \tab}
\newcommand\sameas{\tab \equiv \tab}

\title{CSC263 - Week 2, Lecture 1}
\author{Cristyn Howard}
\date{Monday, January 15, 2018}

\begin{document}
\maketitle


\begin{center}
\begin{tabular}{|c|c|c|c|c|c|}
\hline
ADT & Data Structure & INSERT & MIN & EXTRACTMIN & MERGE	\\
\hline
Priority Queues & Heaps & yes & yes & yes & NO \\
\hline
Mergeable Priority Queues & Binomial Queues & $O(\log{n})$ & $O(\log{n})$ & $O(\log{n})$ & $O(\log{n})$ \\
\hline
\end{tabular}
\end{center}
\vspace{0.4cm}

If a CPU has two cores, each with a Priority Queue of tasks to perform, it might need to merge them. This is not easy to accomplish with Heaps, however it is possible with...
\vspace{0.4cm}

\underline{Binomial Heaps}
\begin{itemize}
\item [] $S_k \: tree: \tab S_0 = \bigcirc$ \newline
	\tab[1.75cm] $S_k$ = take two $S_{k-1}$ trees, make the roof of one the parent of the root of the other.
	
\item [] \begin{center}
\begin{tabular}{|c|c|c|c|c|}
\hline
$S_0$ & $S_1$ & $S_2$ & $S_3$ & $S_4$ \\
\hline
\makecell{$\bigcirc$} & 
\makecell{ \Tree[.$\bigcirc$  $\bigcirc$  ] } & 
\makecell{ \Tree[.$\bigcirc$ [.$\bigcirc$ $\bigcirc$ ] $\bigcirc$ ] } & 
\makecell{\Tree[.$\bigcirc$ [.$\bigcirc$ [.$\bigcirc$ $\bigcirc$ ] $\bigcirc$ ][.$\bigcirc$ $\bigcirc$ ] $\bigcirc$ ] } & 
\makecell{\Tree[.$\bigcirc$ [.$\bigcirc$ [.$\bigcirc$ [.$\bigcirc$ $\bigcirc$ ] $\bigcirc$ ][.$\bigcirc$ $\bigcirc$ ] $\bigcirc$ ] [.$\bigcirc$ [.$\bigcirc$ $\bigcirc$ ] $\bigcirc$ ][.$\bigcirc$ $\bigcirc$ ] $\bigcirc$ ]}  \\
\hline
\end{tabular}
\end{center}

\item Each $S_k$ tree is attached to $\{S_0, S_1, ... , S_{k-1}\}$, has $2^k$ total nodes, and $\binom{k}{d}$ nodes at depth d.
\end{itemize} \vspace{0.4cm}

A \underline{binomial forest} of size $m$, denoted $F_m$, is a sequence of $S_k$ trees with increasing k, and a total of $m$ nodes in the entire forest. 
	\begin{itemize}
	\item Represent the number $m$ in binary, each 1 digit in the resulting number represents an Sk tree in the forest. \framebox{Ex: $m=7=<1,1,1>_2 = 2^2+2^1+2^0 \thfr F_m = \{ S_2, S_1, S_0\}$ }
	\item $\alpha(m) =$ number of 1's in the binary representation of m.
	\item $F_m$ has $\alpha(m)$ trees, and $m-\alpha(m)$ edges.
	\item Note: a 'forest' is just the structure, it becomes a heap when keys are added.
	\end{itemize}
	
\underline{Min heap} has property that the key of the parent is SMALLER than the key of the children.
\newpage

Some examples of binary forests with keys that conform to the min heap property: \newline

\begin{center}
\begin{tabular}{ p{12em} | p{12em}}
\hline
$m=7=<1,1,1>_2$ \newline
$F_m = \{ S_2, S_1, S_0\}$\newline
$S=\{10, 13, 1, 3, 8, 18, 7\}$	& 
\Tree[.$3$ [.$7$ $18$ ] $8$ ] \tab
\Tree[.$1$  $13$  ]  \tab \Tree[.$10$ ] \\
\hline
$m=0=<1,0,0,1>_2$ \newline\newline
$F_m = \{ S_3, S_0\}$\newline\newline
$S=\{6,1,7,3,4,5,2,2,21\}$	& 
\Tree[.$1$ [.$3$ [.$7$ $21$ ] $6$ ][.$2$ $5$ ] $4$ ]
 \tab \Tree[.$2$ ] \\
\hline

\end{tabular}
\end{center}
\vspace{0.4cm}
These forests must have pointers to allow for navigation between trees and interaction with the data structure. These pointers are NOT edges in the trees. \\ \\
Each node stores pointers to: parent, left child, right sibling.


\end{document}