\documentclass[12pt]{article} 
\usepackage[letterpaper, margin=0.5in]{geometry}        		
\geometry{letterpaper}
\usepackage[parfill]{parskip} 
\usepackage{framed}
\usepackage{graphicx}
\usepackage{amsmath}
\usepackage{amssymb}
\usepackage{qtree}
\usepackage{makecell}
\usepackage{lmodern}
\renewcommand*\familydefault{\sfdefault}
\usepackage{tikz}
\usetikzlibrary{matrix}

\usepackage{mathtools}
\DeclarePairedDelimiter\ceil{\lceil}{\rceil}
\DeclarePairedDelimiter\floor{\lfloor}{\rfloor}
\DeclarePairedDelimiter\abs{\lvert}{\rvert}%
\DeclarePairedDelimiter\norm{\lVert}{\rVert}%
    % \abs & \norm resizes brackets, starred version doesn't
    \makeatletter
    \let\oldabs\abs
    \def\abs{\@ifstar{\oldabs}{\oldabs*}}
    %
    \let\oldnorm\norm
    \def\norm{\@ifstar{\oldnorm}{\oldnorm*}}
    \makeatother

\newcommand\graytag[1]{\text{\textsl{\color{gray}{#1}}}}
\newcommand\tab[1][0.4cm]{\hspace*{#1}}
\newcommand\imp{\rightarrow}
\newcommand\thfr{\tab \therefore \tab}
\newcommand\sameas{\tab \equiv \tab}

\renewcommand{\arraystretch}{1.5}

\graphicspath{{../pdf/}{D:\Users\elizabethhoward\Documents\Courses\Current\csc263\extras\wk-2-binomial-heaps\figures}}

\title{CSC263 - Week 5, Lecture 1}
\author{Cristyn Howard}
\date{Monday, February 5, 2018}

\begin{document}
\maketitle

\texttt{Q: A webserver needs to reference a large list of blacklisted websites. The entire list cannot be stored in main memory, because it is too large. How can we store it? \\ \\
	A: In a BLOOM FILTER}
	
\subsection*{Bloom Filters}

\begin{itemize}
\item probablistic (i.e. approximate) dictionary that stores $F_S:$ a summary/fingerprint of a dynamic set S
	\begin{center}\graytag{
	\begin{tabular}{r@{$\:\: : \:\:$}l}
  	BF$[0, ..., m-1]$    & array of $m$ bits, initialized to $0$. \\
  	$\{h_1, ..., h_t \}$  & set of $t<m$ hash fuctions where $h_i:U\rightarrow\{0, ..., m-1\}$     \\
	$m, \:\: t$ & parameters defining $\#$ of bits $\&$ $\#$ of hash functions, respectively
	\end{tabular}}
	\end{center}
	
\item \underline{Operations}: 
	\begin{itemize}
	\item [\graytag{$\Theta(n)$}] INSERT($F_S\:, \: x$) $: $ \graytag{for $i$ from $1$ to $t, \: BF[h_i(x)]=1;$}
		\begin{itemize}
		\item element passed to all hash functions, produces a $t$-tuple of indices $\in \{0, ..., m-1\}$ that form fingerprint of $x$
		\item bits at all indices in fingerprint of $x$ permanently set to $1$
		\end{itemize}
	\item  [\graytag{$\Theta(n)$}] SEARCH($F_S\:, \: x$) $: $ \graytag{for $i$ from $1$ to $t$, if (BF$[h_i(x)] == 0$) return FALSE, else return TRUE}
		\begin{itemize}
		\item Returns: $\{\text{FALSE}\equiv\text{x definitely not in }F_S\} 
			||  \{\text{TRUE}\equiv\text{x \textbf{probably} in }F_S\}$
		\item If any index in the hash of $x$ is $0$, then $x$ has definitely not been placed in $S$.
		\item However, if $x$ is not in $S$, and the union of all indices in the hashes of all elements stored in $S$ contains all indices in the hash of $x$, then SEARCH will return a \textbf{false positive}.
		\end{itemize}
	\end{itemize}
\item \underline{Benefits}: very space and time efficient way to store and search large sets.	
\item \underline{Limitations}: 
	\begin{itemize}
	\item cannot delete elements from S
	\item SEARCH may return \emph{false positives} - may say that $x$ is likely in $S$ when it is not.
	\end{itemize}
\item \underline{Use when}: space constraints exist, false positives are acceptable, and when deleting is not necessary.
\end{itemize}
\newpage \color{white}{.} \color{black} 

\subsubsection*{False Positive Probability Analysis}

Assume $n$ elements inserted. Compute probability that SEARCH(BF, x) returns a false positive, i.e. a positive when x has not been inserted.

Recall that hash functions have uniform, independent probability distributions.





% TODO : FINISH INCOMPLETE NOTE TYPESETTING %






\end{document}