\documentclass[11pt, oneside]{article} 
\usepackage[letterpaper, margin=0.5in]{geometry}        		
\geometry{letterpaper}
\usepackage[parfill]{parskip} 
\usepackage{graphicx}
\usepackage{amssymb}
\usepackage{qtree}
\usepackage{makecell}

\usepackage{mathtools}
\DeclarePairedDelimiter\ceil{\lceil}{\rceil}
\DeclarePairedDelimiter\floor{\lfloor}{\rfloor}
\DeclarePairedDelimiter\abs{\lvert}{\rvert}%
\DeclarePairedDelimiter\norm{\lVert}{\rVert}%
    % \abs & \norm resizes brackets, starred version doesn't
    \makeatletter
    \let\oldabs\abs
    \def\abs{\@ifstar{\oldabs}{\oldabs*}}
    %
    \let\oldnorm\norm
    \def\norm{\@ifstar{\oldnorm}{\oldnorm*}}
    \makeatother
    
\newcommand\tab[1][0.5cm]{\hspace*{#1}}


\title{CSC263 - Week 1, Tutorial 1}
\author{Cristyn Howard}
\date{Friday, January 12, 2018}

\begin{document}
\maketitle

\begin{center} 
\underline{\emph{{\Large Today's topic: Review of running time, proofs of running time.}}} 
\end{center}
\vspace{1em}

\begin{itemize}
\item T(n); worst-case running time; the maximum number of steps the algorithm T takes on an input of size n
	\begin{itemize}
	\item t(x); the number of steps taken by the algorithm on \underline{specific input} x
	\item $T(n) = max\{ t(x) \: \: \mid \: \: \:  \abs{x} = n\}$
	\end{itemize}

\item $O, \Omega, \Theta$; mathematical concepts that quantify the relations between functions; 
   	\begin{itemize}
   	\item adopted by computer scientists to abstractly compare running time of different algorithms, without concern for specific implementation details 
	\item $O(n^2) = \{1, 2, n^2, 2n^2, n^2+1, ...\}$, (infinite set)
	\item $\Omega(n^2) = \{ n^2, 2n^2, 4n^3, ...\}$, (infinite set)
	\item $\Theta(n^2) = O(n^2) \cap \Omega(n^2) = \{n^2, 2n^2, ...\}$, (infinite set)
    	\end{itemize}

\item $T(n) \in O(g(n)) \iff $ for \underline{every} input of size $n$, T takes \underline{at most} $c\times g(n)$ steps, (where $c \in \mathbb{R}$)

\item $T(n) \in \Omega(g(n)) \iff $ there is \underline{some} input of size $n$ for which T takes \underline{at least} $c\times g(n)$ steps, (where $c \in \mathbb{R}$)
\vspace{1em}

\item [Ex 1:] 
	\begin{tabular}{  l | l  } 
	\makecell[l]{		input: A[1...n] \\ 
				for $i=1$ to n: \\ 
				\tab for $j=1$ to n: \\
				\tab \tab if A[i] != 1, STOP}
	&	
	\makecell[l]{	$O(n^2)$ because the max possible iterations of the inner $\&$ outer loop \\
				result in $n^2$ calls of the fourth line of code, which is in constant time \\
				\\
				$\Omega(n^2)$ because $\forall \: \: n, \exists$ array of 1's of size n, \\
				which will result in runtime of $n^2$ } 
	\end{tabular}
\vspace{1em}

\item [Ex 2:] 
	\begin{tabular}{  l | l  } 
	\makecell[l]{
		Bubblesort(A[1...n]): \\
		last = n, sorted = false; \\
		while(not sorted): \\
		\tab sorted = true; \\
		\tab for j=1 to last-1: \\
		\tab \tab if (A[j] $>$ A[j+1]): \\
		\tab \tab \tab swap A[j] $\&$ A[j+1]; \\
		\tab \tab \tab sorted = false; \\
		\tab last -= 1; \\ }
	&	
	\makecell[l]{ 
		\underline{Loop Invariant}: at the end of the $i^{th}$ iteration of the while loop, \\
		A[last+1 ... n] contains the largest elements in A in sorted order \\ \\
		$T(n) \in O(n^2)$: \\
		\tab $\bullet$ while loop executes at most n times \\
		\tab \tab (each loop reduces 'last' by 1; when 'last'=1, sorted set true, \\
		\tab \tab for loop not executed, while loop stops) \\
		\tab $\bullet$ inner loop executes at most n-1 times \\
		\tab $\bullet$ $n (n-1) \approx n^2 $ \\ \\
		$T(n) \in \Omega(n^2)$: \\
		\tab $\bullet$ consider reverse sorted array of size n \\
		\tab $\bullet$ for $1^{st}$ iteration of while loop, for loop does n-1 swaps \\ 
		\tab $\bullet$ for $i^{th}$ iteration of while loop, for loop does n-i swaps \\ 
		\tab $\bullet$ total swaps reverse sorted array of size n = $\sum_{i=1}^{n} n-i = \frac{n(n-1)}{2} \in \Theta(n^2)$ \\
		\tab $\bullet$ thus for every n, $\exists$ an input array for which the number of steps is \\
		\tab \tab at least $\in \Theta(n^2)$} 
	\end{tabular}

\end{itemize}










\end{document}  