\documentclass[12pt]{article} 
\usepackage[letterpaper, margin=0.5in]{geometry}        		
\geometry{letterpaper}
\usepackage[parfill]{parskip} 
\usepackage{framed}
\usepackage{graphicx}
\usepackage{amsmath}
\usepackage{amssymb}
\usepackage{qtree}
\usepackage{makecell}
\usepackage{lmodern}
\renewcommand*\familydefault{\sfdefault}
\usepackage{tikz}
\usetikzlibrary{matrix}

\usepackage{mathtools}
\DeclarePairedDelimiter\ceil{\lceil}{\rceil}
\DeclarePairedDelimiter\floor{\lfloor}{\rfloor}
\DeclarePairedDelimiter\abs{\lvert}{\rvert}%
\DeclarePairedDelimiter\norm{\lVert}{\rVert}%
    % \abs & \norm resizes brackets, starred version doesn't
    \makeatletter
    \let\oldabs\abs
    \def\abs{\@ifstar{\oldabs}{\oldabs*}}
    %
    \let\oldnorm\norm
    \def\norm{\@ifstar{\oldnorm}{\oldnorm*}}
    \makeatother

\newcommand\graytag[1]{\text{\textsl{\color{gray}{#1}}}}
\newcommand\tab[1][0.5cm]{\hspace*{#1}}
\newcommand\imp{\rightarrow}
\newcommand\thfr{\tab \therefore \tab}
\newcommand\sameas{\tab \equiv \tab}

\title{CSC263 - Week 3, Lecture 1}
\author{Cristyn Howard}
\date{Monday, January 22, 2018}

\begin{document}
\maketitle

\begin{center}
\begin{tabular}{|c|c| c |}
\hline
ADT & Operations & Data Structure 	\\
\hline
Dictionary & Insert, Delete, Search  & 
\makecell[l]{BST\\Balanced Trees \{ 2-3 trees, B-trees, red-black trees, AVL trees \} } \\
\hline
\end{tabular}
\end{center}
\vspace{0.4cm}

\underline{Problems with BSTs}...
\begin{center}
\begin{tabular}{ c c c c c }

suppose we start with... & \Tree[.$(2)$ $(1)$ $(3)$ ] &
and add $4, 5, 6...$ & \Tree[.$(2)$ [.$(1)$ ][.$(3)$ [.$(4)$ [.$(5)$ $(6)$ ]] ] ]
\end{tabular}
\end{center}

Unbalanced trees such as the one above results in height $\in \Theta(n)$, ops  $\in \Theta(n)$. Today we will explore a new data structure, balanced trees. First, some definitions.

\begin{itemize}
\item Height(v): $\#$ of edges on the longest path from node v to a leaf
\item BalanceFactor(v): Height(v.rightSubTree) $-$ Height(v.leftSubTree)
\end{itemize}

AVL $ $




\end{document}