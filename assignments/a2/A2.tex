\documentclass[11pt]{article} 
\usepackage[letterpaper, margin=0.5in]{geometry}        		
\geometry{letterpaper}
\usepackage[parfill]{parskip} 
\usepackage{framed}
\usepackage{graphicx}
\usepackage{amsmath}
\usepackage{amssymb}
\usepackage{qtree}
\usepackage{makecell}
\usepackage{lmodern}
\renewcommand*\familydefault{\sfdefault}
\usepackage{tikz}
\usetikzlibrary{matrix}

\usepackage{mathtools}
\DeclarePairedDelimiter\ceil{\lceil}{\rceil}
\DeclarePairedDelimiter\floor{\lfloor}{\rfloor}
\DeclarePairedDelimiter\abs{\lvert}{\rvert}%
\DeclarePairedDelimiter\norm{\lVert}{\rVert}%
    % \abs & \norm resizes brackets, starred version doesn't
    \makeatletter
    \let\oldabs\abs
    \def\abs{\@ifstar{\oldabs}{\oldabs*}}
    %
    \let\oldnorm\norm
    \def\norm{\@ifstar{\oldnorm}{\oldnorm*}}
    \makeatother

\newcommand\graytag[1]{\text{\textsl{\color{gray}{#1}}}}
\newcommand\tab[1][0.5cm]{\hspace*{#1}}
\newcommand\imp{\rightarrow}
\newcommand\thfr{\tab \therefore \tab}
\newcommand\sameas{\tab \equiv \tab}

\graphicspath{{../pdf/}{D:\Users\elizabethhoward\Documents\Courses\Current\csc263\extras\wk-2-binomial-heaps\figures}}

\title{CSC263 - Assignment 2}
\author{Cristyn Howard}
\date{Monday, January 29, 2018}

\begin{document}
\maketitle

\subsection*{Question 1}
\subsubsection*{Part a)}
\begin{itemize}
\item Let $\alpha(n)$ denote the number of $1$'s in the binary representation of n.
\item \underline{Lemma}: based on the structural definition of binary forests, a binary forest with $n$ nodes contains $\alpha(n)$ trees.
\item \underline{Lemma}: a tree with $n$ nodes contains $n-1$ edges.
\item Let the trees in a binary heap with $n$ nodes be labelled $G_i$, where $1 \leq i \leq \alpha(n)$, and let  $k_i$ denote the number of nodes in tree $G_i$.
\item We know that $\sum_{i=1}^{\alpha(n)}k_i = n$, so then $\sum_{i=1}^{\alpha(n)} (ki-1) = n - \alpha(n)$.
\item Thus we can conclude that a binary heap with $n$ nodes contains $ n - \alpha(n)$ edges.
\end{itemize}

\subsubsection*{Part b)}
\begin{itemize}
\item Binomial heap H has $n$ nodes and $n-\alpha(n)$ edges before any insertions.
\item After $k$ insertions, H has $n+k$ nodes and $(n+k)-\alpha(n+k)$ edges.
\item The number of edges created by $k$ insertions is thus $[(n+k)-\alpha(n+k)] - [n-\alpha(n)] = n+k-\alpha(n+k)-n+\alpha(n) = k + \alpha(n) - \alpha(n+k)$.
	\begin{itemize}
	\item Note that $\alpha(n+k)\geq 0$, and so $k + \alpha(n) - \alpha(n+k) \leq k + \alpha(n)$.
	\item Further, note that $\alpha(n) \leq \floor{\log_{2}{n}}+1$ (by the definition of $\alpha(n)$).
	\item So all together we have $k + \alpha(n) - \alpha(n+k) \leq k+ \floor{\log_{2}{n}}+1$.
	
	\end{itemize}
\item Pairwise comparisons between node values (which occur in constant time) take place whenever an insertion of an element into H creates a new edge, as the newly connected nodes must be compared and potentially swapped to preserve the heap order property. 
\item The dominant factor affecting the worst case running time of inserting $k$ elements into H is the number of comparisons that must occur, or the number of edges that must be created, which we know is bounded above by $k+ \floor{\log_{2}{n}}+1$.
\item When $k<\log_{2}{n}$, the dominant term in $[k+ \floor{\log_{2}{n}}+1]$ is $\log_{2}{n}$, so the worst case running cost of inserting $k$ elements into H is $O(\log_{2}{n})$.
\item However when $k>\log_{2}{n}$, the dominant term in $[k+ \floor{\log_{2}{n}}+1]$ is $k$, so the worst case running cost of inserting $k$ elements into H is $O(k)$, and the average cost of inserting $1$ element is $O(k/k)$, or $O(1)$.
\end{itemize}

\subsection*{Question 2}
\begin{center}
\includegraphics[totalheight=15cm]{pseudocode-wide.png}
\end{center}

\begin{itemize}
\item The wrapper function contains only constant time operations (besides the call to check-bal).
\item Base case inputs to check-bal that trigger no recursive calls ($x=$ leaf, $x=$ NIL) contain only constant time operations.
\item In non-base cases (when $x$ has at least one child), all operations that are not recursive calls (evaluating if conditions, doing arithmetic operations, executing return statements) occur in constant time.
\item Then the dominant factor in the running time is the number of recursive calls.
\item Whenever $x$ has one child, check-bal is called on that child, and whenever $x$ has two children, check-bal is structured to call on both of them. So check-bal is structured to call recursively on all nodes of the subtree rooted at $x$. 
\item When $x$ is the root of a 2-balanced BST (no recursive calls to check-bal return FALSE), we know that check-bal is called recursively \underline{on all nodes} of the BST, thus there exists an input $\forall n \in \mathbb{N}$ where check-bal is called $n$ times. So $T \in \Omega(n)$.
\item Check-bal is called \underline{at most} once per node, therefore the maximum number of check-bal calls for a BST of size $n \in \mathbb{N}$ is $n$. So $T \in O(n)$.
\item Thus the worst case running time of this algorithm is $\Theta(n)$, where $n$ is the $\#$ of nodes in the BST rooted at $u$.


\end{itemize}

\newpage

\end{document}